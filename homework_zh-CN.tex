\documentclass{homework}

% 可选的版式/字体配置示例(默认关闭,按需取消注释):
% \homeworksetlinespread{1.2}
% \homeworksetmargins{left=2.6cm,right=2.6cm,top=2.2cm,bottom=2.4cm}
% \homeworksetproblemtitlespacing{*1.8}{*1}
% \homeworksetsolutionskip{0.6\baselineskip}{0.5\baselineskip}
% \homeworksetCJKmainfont{Source Han Serif SC}
% \homeworksetmainfont{Times New Roman}
% \homeworksetmathfont{XITS Math}

\renewcommand{\hmwkTitle}{第 1 次作业}
\renewcommand{\hmwkDueDate}{2025 年 3 月 15 日}
\renewcommand{\hmwkClass}{数值分析}
\renewcommand{\hmwkClassTime}{周三 3-4 节}
\renewcommand{\hmwkClassInstructor}{张老师}
\renewcommand{\hmwkAuthorName}{张三}

% 如需将“Solution”标题替换为中文,可按需启用:
% \renewcommand{\solution}{\par\vspace{\hmwkSolutionBeforeSkip}\textbf{解答}\par\vspace{\hmwkSolutionAfterSkip}}

\begin{document}

\maketitle

\pagebreak

\begin{homeworkProblem}
    设存在正的常数 \(c\),使得当 \(n>1\) 时 \(f(n) \leq c\cdot g(n)\)。
    求满足条件的常数 \(c\)。

    \begin{enumerate}
        \item \(f(n) = n^2 + n + 1\),\(g(n) = 2n^3\)
        \item \(f(n) = n\sqrt{n} + n^2\),\(g(n) = n^2\)
        \item \(f(n) = n^2 - n + 1\),\(g(n) = n^2 / 2\)
    \end{enumerate}

    \solution

    对每一小问分别估计上界即可求得一个可行的 \(c\)。

    \textbf{小问一}

    \[
        \begin{split}
            n^2 + n + 1
            &\leq n^2 + n^2 + n^2\\
            &= 3n^2\\
            &\leq 2c\, n^3
        \end{split}
    \]

    取 \(c = 2\) 即可满足条件。

    \textbf{小问二}

    \[
        \begin{split}
            n^2 + n\sqrt{n}
            &= n^2 + n^{3/2}\\
            &\leq n^2 + n^{4/2}\\
            &= 2n^2\\
            &\leq c\, n^2
        \end{split}
    \]

    取 \(c = 2\) 即可。

    \textbf{小问三}

    \[
        \begin{split}
            n^2 - n + 1
            &\leq n^2\\
            &\leq \tfrac{1}{2}c\, n^2
        \end{split}
    \]

    仍可令 \(c = 2\)。
\end{homeworkProblem}

\pagebreak

\begin{homeworkProblem}
    证明多项式
    \(P(n) = a_kn^k + a_{k-1}n^{k-1} + \cdots + a_0\)
    (其中 \(a_i \geq 0\))属于 \(\Theta(n^k)\)。

    \solution

    因为每个系数均非负,故有
    \(P(n) \geq a_kn^k\),于是 \(P(n) = \Omega(n^k)\)。
    另一方面,利用多项式的单调性可得

    \[
        P(n)
        \leq (a_k + a_{k-1} + \cdots + a_0) n^k,
    \]

    从而存在常数 \(c_2 = a_k + \cdots + a_0\),使得
    \(P(n) \leq c_2 n^k\),即 \(P(n) = O(n^k)\)。
    结合上下界可知 \(P(n) = \Theta(n^k)\)。
\end{homeworkProblem}

\end{document}
